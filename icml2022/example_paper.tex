%%%%%%%% ICML 2022 EXAMPLE LATEX SUBMISSION FILE %%%%%%%%%%%%%%%%%

\documentclass[nohyperref]{article}

% Recommended, but optional, packages for figures and better typesetting:
\usepackage{microtype}
\usepackage{graphicx}
\usepackage{subfigure}
\usepackage{booktabs} % for professional tables

% hyperref makes hyperlinks in the resulting PDF.
% If your build breaks (sometimes temporarily if a hyperlink spans a page)
% please comment out the following usepackage line and replace
% \usepackage{icml2022} with \usepackage[nohyperref]{icml2022} above.
\usepackage{hyperref}


% Attempt to make hyperref and algorithmic work together better:
\newcommand{\theHalgorithm}{\arabic{algorithm}}

% Use the following line for the initial blind version submitted for review:
\usepackage{icml2022}

% If accepted, instead use the following line for the camera-ready submission:
% \usepackage[accepted]{icml2022}

% For theorems and such
\usepackage{amsmath}
\usepackage{amssymb}
\usepackage{mathtools}
\usepackage{amsthm}

% if you use cleveref..
\usepackage[capitalize,noabbrev]{cleveref}

%%%%%%%%%%%%%%%%%%%%%%%%%%%%%%%%
% THEOREMS
%%%%%%%%%%%%%%%%%%%%%%%%%%%%%%%%
\theoremstyle{plain}
\newtheorem{theorem}{Theorem}[section]
\newtheorem{proposition}[theorem]{Proposition}
\newtheorem{lemma}[theorem]{Lemma}
\newtheorem{corollary}[theorem]{Corollary}
\theoremstyle{definition}
\newtheorem{definition}[theorem]{Definition}
\newtheorem{assumption}[theorem]{Assumption}
\theoremstyle{remark}
\newtheorem{remark}[theorem]{Remark}

% Todonotes is useful during development; simply uncomment the next line
%    and comment out the line below the next line to turn off comments
%\usepackage[disable,textsize=tiny]{todonotes}
\usepackage[textsize=tiny]{todonotes}


% The \icmltitle you define below is probably too long as a header.
% Therefore, a short form for the running title is supplied here:
\icmltitlerunning{Submission and Formatting Instructions for ICML 2022}

\begin{document}

\twocolumn[
\icmltitle{Submission and Formatting Instructions for \\
           International Conference on Machine Learning (ICML 2022)}

% It is OKAY to include author information, even for blind
% submissions: the style file will automatically remove it for you
% unless you've provided the [accepted] option to the icml2022
% package.

% List of affiliations: The first argument should be a (short)
% identifier you will use later to specify author affiliations
% Academic affiliations should list Department, University, City, Region, Country
% Industry affiliations should list Company, City, Region, Country

% You can specify symbols, otherwise they are numbered in order.
% Ideally, you should not use this facility. Affiliations will be numbered
% in order of appearance and this is the preferred way.
\icmlsetsymbol{equal}{*}

\begin{icmlauthorlist}
\icmlauthor{Firstname1 Lastname1}{equal,yyy}
\icmlauthor{Firstname2 Lastname2}{equal,yyy,comp}
\icmlauthor{Firstname3 Lastname3}{comp}
\icmlauthor{Firstname4 Lastname4}{sch}
\icmlauthor{Firstname5 Lastname5}{yyy}
\icmlauthor{Firstname6 Lastname6}{sch,yyy,comp}
\icmlauthor{Firstname7 Lastname7}{comp}
%\icmlauthor{}{sch}
\icmlauthor{Firstname8 Lastname8}{sch}
\icmlauthor{Firstname8 Lastname8}{yyy,comp}
%\icmlauthor{}{sch}
%\icmlauthor{}{sch}
\end{icmlauthorlist}

\icmlaffiliation{yyy}{Department of XXX, University of YYY, Location, Country}
\icmlaffiliation{comp}{Company Name, Location, Country}
\icmlaffiliation{sch}{School of ZZZ, Institute of WWW, Location, Country}

\icmlcorrespondingauthor{Firstname1 Lastname1}{first1.last1@xxx.edu}
\icmlcorrespondingauthor{Firstname2 Lastname2}{first2.last2@www.uk}

% You may provide any keywords that you
% find helpful for describing your paper; these are used to populate
% the "keywords" metadata in the PDF but will not be shown in the document
\icmlkeywords{Machine Learning, ICML}

\vskip 0.3in
]

% this must go after the closing bracket ] following \twocolumn[ ...

% This command actually creates the footnote in the first column
% listing the affiliations and the copyright notice.
% The command takes one argument, which is text to display at the start of the footnote.
% The \icmlEqualContribution command is standard text for equal contribution.
% Remove it (just {}) if you do not need this facility.

%\printAffiliationsAndNotice{}  % leave blank if no need to mention equal contribution
\printAffiliationsAndNotice{\icmlEqualContribution} % otherwise use the standard text.


\begin{abstract}
    This report presents a study on supply chain management using Deep Q-Network (DQN) reinforcement learning. We investigate the performance of a single-agent DQN baseline algorithm in optimizing the profit of a specific firm within a multi-firm supply chain environment. The environment simulates a supply chain with multiple firms, where each firm's demand is generated using a Poisson distribution. The DQN agent is trained to maximize its profit by making optimal ordering decisions, while other firms follow either random or classical supply chain management strategies. Our results demonstrate the effectiveness of the DQN approach in improving the profit of the trained firm compared to traditional strategies.
    \end{abstract}

    \section{Introduction}
    Supply chain management is a critical area in operations research and management science. The ability to optimize ordering decisions can significantly impact a firm's profitability. In this study, we explore the application of reinforcement learning, specifically Deep Q-Network (DQN), to optimize ordering decisions in a supply chain environment. We focus on a single-agent approach where one firm is trained using DQN, while other firms follow predefined strategies.
    
    \section{Methodology}
    \subsection{Environment}
    The supply chain environment is simulated with multiple firms, each characterized by parameters such as price, holding cost, and lost sales cost. The demand for the downstream firm is generated using a Poisson distribution, while the demand for upstream firms is determined by the orders placed by downstream firms.
    
    \subsection{Deep Q-Network (DQN)}
    The DQN algorithm is used to train a single agent (firm) to make optimal ordering decisions. The agent's state space includes the current order, satisfied demand, and inventory levels. The action space is discrete, representing the order quantity. The reward is defined as the profit, which is calculated based on the revenue from satisfied demand, the cost of orders, and the holding cost of inventory.
    
    \subsection{Training Process}
    The training process involves multiple episodes, where each episode consists of a series of steps. The agent interacts with the environment, observes the state, takes an action, receives a reward, and updates its policy using the DQN algorithm. The exploration rate is gradually decreased to balance exploration and exploitation.
    
    \section{Results}
    \subsection{Training Performance}
    The training performance is evaluated based on the cumulative reward (profit) achieved by the DQN agent over episodes. The results show that the DQN agent improves its performance over time, achieving higher profits compared to random strategies.
    
    \subsection{Testing Performance}
    The testing performance is evaluated by running the trained agent in the environment for a fixed number of episodes. The results demonstrate the agent's ability to make optimal ordering decisions, leading to higher profits compared to other firms following random or classical strategies.
    
    \section{Discussion}
    The results indicate that the DQN approach is effective in optimizing the profit of the trained firm. The agent learns to balance the trade-off between ordering too much (increasing holding costs) and ordering too little (increasing lost sales costs). However, the performance of the DQN agent may vary depending on the strategies followed by other firms in the supply chain.
    
    \section{Conclusion}
    In conclusion, the application of DQN in supply chain management shows promising results in optimizing the profit of a single firm. Future work could explore multi-agent DQN approaches to optimize the entire supply chain's performance.
    
    \section{Future Work}
    Future research could focus on:
    \begin{itemize}
        \item Extending the DQN approach to multi-agent scenarios.
        \item Investigating the impact of different demand distributions on the performance of the DQN agent.
        \item Exploring the use of other reinforcement learning algorithms, such as Policy Gradient or Actor-Critic methods.
    \end{itemize}
    


% In the unusual situation where you want a paper to appear in the
% references without citing it in the main text, use \nocite
\nocite{langley00}

\bibliography{example_paper}
\bibliographystyle{icml2022}


%%%%%%%%%%%%%%%%%%%%%%%%%%%%%%%%%%%%%%%%%%%%%%%%%%%%%%%%%%%%%%%%%%%%%%%%%%%%%%%
%%%%%%%%%%%%%%%%%%%%%%%%%%%%%%%%%%%%%%%%%%%%%%%%%%%%%%%%%%%%%%%%%%%%%%%%%%%%%%%
% APPENDIX
%%%%%%%%%%%%%%%%%%%%%%%%%%%%%%%%%%%%%%%%%%%%%%%%%%%%%%%%%%%%%%%%%%%%%%%%%%%%%%%
%%%%%%%%%%%%%%%%%%%%%%%%%%%%%%%%%%%%%%%%%%%%%%%%%%%%%%%%%%%%%%%%%%%%%%%%%%%%%%%
\newpage
\appendix
\onecolumn
\section{You \emph{can} have an appendix here.}

You can have as much text here as you want. The main body must be at most $8$ pages long.
For the final version, one more page can be added.
If you want, you can use an appendix like this one, even using the one-column format.
%%%%%%%%%%%%%%%%%%%%%%%%%%%%%%%%%%%%%%%%%%%%%%%%%%%%%%%%%%%%%%%%%%%%%%%%%%%%%%%
%%%%%%%%%%%%%%%%%%%%%%%%%%%%%%%%%%%%%%%%%%%%%%%%%%%%%%%%%%%%%%%%%%%%%%%%%%%%%%%


\end{document}


% This document was modified from the file originally made available by
% Pat Langley and Andrea Danyluk for ICML-2K. This version was created
% by Iain Murray in 2018, and modified by Alexandre Bouchard in
% 2019 and 2021 and by Csaba Szepesvari, Gang Niu and Sivan Sabato in 2022. 
% Previous contributors include Dan Roy, Lise Getoor and Tobias
% Scheffer, which was slightly modified from the 2010 version by
% Thorsten Joachims & Johannes Fuernkranz, slightly modified from the
% 2009 version by Kiri Wagstaff and Sam Roweis's 2008 version, which is
% slightly modified from Prasad Tadepalli's 2007 version which is a
% lightly changed version of the previous year's version by Andrew
% Moore, which was in turn edited from those of Kristian Kersting and
% Codrina Lauth. Alex Smola contributed to the algorithmic style files.
